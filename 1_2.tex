\documentclass{oxmathproblems}

\printanswers

\course{Understanding Analysis, 2nd Edition: Stephen Abbott}
\sheettitle{Chapter 1 - Section 1.2} %can leave out if no title per sheet

\begin{document}
\begin{questions}

%------------------------- Problem 1 -------------------------
\miquestion Prove that $\sqrt{3}$ is irrational. Does the same proof work for $\sqrt{6}$ and $\sqrt{4}$.
\begin{solution}
  Rather than a proof by contradiction like in the book, a direct proof is more illuminating, although we need to cheat a bit by using unique prime factorization. 
  
  Let $(\frac{p}{q})^{2} = 3$. Since the LHS is a square, it must have an even number of prime factors, while the RHS has an ood number of factors, qed.

  This makes it immediately obvious that the proof works for any integer with an odd number of factors and conversely, it doesn't work for an integer with an even
  number of factors.
\end{solution}

%------------------------- Problem 2 -------------------------
\miquestion Show that there is no rational number satisfying $2^{r} = 3$.
\begin{solution}
  There is no solution to $2^{p} = 3^{q}$ because they have different parities.

  More generally, to check whether $a^{n} = b^{m}$ has a solution, we can look at the prime factorizations. First, the set of prime factors must match.
  Next, if we have $p^{x}$ in $a$ and $p^{y}$ in $b$, then we have $xn = ym \implies \frac{x}{y} = \frac{m}{n}$. This ratio must agree for all prime factors.
\end{solution}

%------------------------- Problem 3 -------------------------
\miquestion See the book.
\begin{solution}
  \begin{parts}
    \part False, as in Example 1.2.2.
    \part True. Note the crucial difference with part (a). In part (a) there's always the "next" set $A_{m+1}$ that doesn't contain $m$. In this part,
    because $A_{1}$ is finite, then either $A_{i+1}=A_{i}$ in which case both contains $m$, or $A_{i+1} \subset A_{i}$ in which case the size of
    intersections decreases by one and since $A_{1}$ is finite and $A_{n}$ not empty, this process must eventually end.
    \part False. Let $A = \{1\}, B = \{2\}$ and $C = \{3\}$.
    \part True.
    \part True.
  \end{parts}
\end{solution}

%------------------------- Problem 4 -------------------------
\miquestion Produce an infinite collection of sets $A_{1}, A_{2}, ...$ with the property that every $A_{i}$ has an infinite number of elements,
$A_{i} \cap A_{j} = \emptyset$ for all $i \neq j$ and $\cup_{i=1}^{\infty}A_{i} = \textbf{N}$.
\begin{solution}
  First thought when I tried to visualize "crossing out integers" is the prime sieve e.g. like Sieve of Eratosthenes.
  So the sets would be $A_{1} = \{2, 4, 6, 8, ...\}, A_{2} = \{3, 9, ...\}$. But there's also 1. If the first set contains
  multiples of 1, then we'd only have a single set instead of infinite collection of sets.

  It took me way too long to realize this but, we can just put 1 in any of the sets e.g. in the set that starts with 3, since 1 doesn't appear anywhere else..
  Now I feel stupid.

  There are many other excellent constructions. Another one I liked (and should have thought of, since it feels more "algorithmic") is to first
  divide by parity, and then divide the even numbers into two sets: one divisible by only 2 and the other divisible by 4, and so on..
  
  A general way to decompose infinite sets is written  \href{https://math.stackexchange.com/a/51097}{here}.
\end{solution}

%------------------------- Problem 5 -------------------------
\miquestion (\textbf{De Morgan's Laws}). Let $A$ and $B$ be subsets of \textbf{R}.
\begin{parts}
  \part If $x \in (A \cap B)^{c}$, explain why $x \in A^{c} \cup B^{c}$. This shows that $(A \cap B)^{c} \subseteq A^{c} \cup B^{c}$.
  \part Prove the reverse inclusion $(A \cap B)^{c} \supseteq A^{c} \cup B^{c}$, and conclude that $(A \cap B)^{c} = A^{c} \cup B^{c}$.
  \part Show $(A \cup B)^{c} = A^{c} \cap B^{c}$ by demonstrating inclusion both ways.
\end{parts} 

\begin{solution}
  Mostly straightforward. Too lazy to type.
\end{solution}

%------------------------- Problem 6 -------------------------
\miquestion Prove a bunch of absolute value stuffs. For all of these, the fastest way to gain an "intuitive" proof is to simply visualize them in the number line.
Too lazy to type the full algebra.

%------------------------- Problem 7 -------------------------
\miquestion Given a function $f$ and a subset $A$ of its domain, let $f(A)$ represent the range of $f$ over $A$; that is, $f(A) = \{f(x): x \in A\}$.
\begin{parts}
  \part Let $f(x) = x^{2}$. If $A = [0, 2]$ and $B = [1, 4]$, find $f(A)$ and $f(B)$. Does $f(A \cap B) = f(A) \cap f(B)$ in this case? Does $f(A \cup B) = f(A) \cup f(B)$?
  \part Find two sets $A$ and $B$ for which $f(A \cap B) \neq f(A) \cap f(B)$.
  \part Show that, for an arbitrary function $g: \textbf{R} \implies \textbf{R}$, it is always true that $g(A \cap B) \subseteq g(A) \cap g(B)$ for all sets $A, B \subseteq \textbf{R}$.
  \part Form an prove a conjecture about the relationship between $g(A \cup B)$ and $g(A) \cup g(B)$ for an arbitrary function $g$.
\end{parts}

\begin{solution}
  \begin{parts}
    \part $f(A) = [0, 4]$ and $f(B) = [1, 16]$. 
    
    $f(A \cap B) = f([1, 2]) = [1, 4]$ and $f(A) \cap f(B) = [1, 4]$ so they're equal.

    $f(A \cup B) = f([0, 4]) = [0, 16]$ and $f(A) \cup f(B) = [0, 16]$ so they're equal.

    \part $A = [-4, -3]$ and $B = [3, 4]$. Then $f(A \cap B) = \emptyset$ but $f(A) \cap f(B) = f(A) = f(B) = [9, 16]$.

    \part If $x \in A \cap B$, then $g(x) \in g(A)$ (because $x \in A$) and similarly $g(x) \in g(B)$, so $g(x) \in g(A) \cap g(B)$.

    \part $g(A \cup B) = g(A) \cup g(B)$. This is not difficult to prove. Too lazy to type.
  \end{parts}
\end{solution}

%------------------------- Problem 8 -------------------------
\miquestion Give an example of each or state that the request is impossible:
\begin{parts}
  \part $f: \textbf{N} \rightarrow \textbf{N}$ that is injective but not surjective.
  \part $f: \textbf{N} \rightarrow \textbf{N}$ that is surjective but not injective.
  \part $f: \textbf{N} \rightarrow \textbf{Z}$ that is injective and surjective.
\end{parts}

\begin{solution}
  \begin{parts}
    \part $f(n) = 2n$. Odd numbers are not in the image of $f$.
    \part $f(n) = floor(n/2)$. Yes, I'm including 0 as a natural number here, sue me.
    \part Split \textbf{N} by parity and \textbf{Z} by sign (with 0 going to the negative subset). 
    We'll map even \textbf{N} to positive \textbf{Z} and odd \textbf{N} to non-positive \textbf{Z}. Define:
    \[
      f(n) =
      \begin{cases}
      \frac{n}{2}, & \text{if } n \text{ is even} \\
      -\left\lceil \frac{n}{2} \right\rceil, & \text{if } n \text{ is odd}
      \end{cases}
    \]
  \end{parts}
\end{solution}

%------------------------- Problem 9 -------------------------
\miquestion Skip. Looks similar to problem 7.

%------------------------- Problem 10 -------------------------
\miquestion Which are true and provide counterexamples if not.
\begin{parts}
  \part Two real numbers satisfy $a < b$ if and only if $a < b + \epsilon$ for every $\epsilon > 0$.
  \part Two real numbers satisfy $a < b$ if $a < b + \epsilon$ for every $\epsilon > 0$.
  \part Two real numbers satisfy $a \leq b$ if and only if $a < b + \epsilon$ for every $\epsilon > 0$.
\end{parts}

\begin{solution}
  \begin{parts}
    \part False. We can have $a=b$.
    \part False. We can have $a=b$.
    \part True.
  \end{parts}
\end{solution}

\miquestion I'm kinda tired of the rest. Maybe some other day...

\end{questions}
\end{document}
