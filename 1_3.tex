\documentclass{oxmathproblems}

\printanswers

\course{Understanding Analysis, 2nd Edition: Stephen Abbott}
\sheettitle{Chapter 1 - Section 1.2} %can leave out if no title per sheet

\begin{document}
\begin{questions}

%------------------------- Problem 1 -------------------------
\miquestion Doesn't look interesting. Skip for now.

%------------------------- Problem 2 -------------------------
\miquestion Give an example or state that it's impossible.
\begin{parts}
  \part A set $B$ with inf $B \geq$ sup $B$.
  \part A finite set that contains its infimum but not its supremum.
  \part A bounded subset of \textbf{Q} that contains its supremum but not its infimum.
\end{parts}

\begin{solution}
  \begin{parts}
    \part A single-element set has its infimum equal supremum.
    \part Slightly tricky. The half-open interval [0, 1) of real numbers is *not* a valid answer, because it's not a finite set.

    The answer is it's impossible. A finite set always has its minimum and maximum as infimum and supremum respectively.

    \part (0, 1].
  \end{parts}
\end{solution}

%------------------------- Problem 3 -------------------------
\miquestion 
\begin{parts}
  \part Let $A$ be nonempty and bounded below, and define $B = \{b \in \textbf{R}: \text{b is a lower bound for A}\}$. Show that sup
  $B = $ inf $A$.
  \part Use (a) to explain why there is no need to assert that greatest lower bounds exist as part of the Axiom of Completeness.
\end{parts}

\begin{solution}
  \begin{parts}
    \part First we need to assert that sup $B$ actually exists. By the axiom completeness, any set bounded above has a supremum, and any element of $A$ is an
    upper bound, so sup $B$ exists.

    Next we prove that sup $B$ = inf $A$. Any real number less than sup $B$ clearly cannot be the infimum, since it wouldn't be the *greatest* lower bound.
    There is also no lower bound greater than sup $B$, since by definition sup $B$ is greatest of all lower bounds. This completes the proof.

    \part Because (a) already shows that the infimum exists (if the set is bounded below) as the supremum of another set, and it also gives the construction of such
    a set (by taking the set of all lower bounds).
  \end{parts}
\end{solution}

%------------------------- Problem 4 -------------------------
\miquestion Let $A_{1}, A_{2}, ..., $ be a collection of nonempty sets, each of which is bounded above.
\begin{parts}
  \part Find a formula for sup($A_{1} \cup A_{2}$). Extend this to sup($\cup^{n}_{k=1}A_{k}$).
  \part Consider sup($\cup^{\infty}_{k=1}A_{k}$). Does the formula in (a) extend to the infinite case?
\end{parts}

\begin{solution}
  \begin{parts}
    \part max(sup $A_{1}$, sup $A_{2}$). max(sup $A_{1}$, ..., sup $A_{n}$). Too lazy to write the formal proof, but it should be intuitively clear.
    \part No, because the resulting union of supremums might not be bounded above. The simplest example is $A_{i}={i}$.
  \end{parts}
\end{solution}

%------------------------- Problem 5 -------------------------
\miquestion As in example example 1.3.7, let $A \subseteq \textbf{R}$ be nonempty and bounded above, and let $c \in \textbf{R}$. This time define the set
$cA = \{ca: a \in A\}$.
\begin{parts}
  \part If $c \geq 0$, show that sup($cA$) = $c$ sup($A$).
  \part Postulate a similar type of statement for sup($cA$) for the case $c < 0$.
\end{parts}

\begin{solution}
  \begin{parts}
    \part $c$ sup $A$ is an upper bound because $a \leq \text{sup} A$ for all $a$ and that implies $ca \leq c \text{ sup } A$ for all $a$.

    It is the \textit{least} upper bound. Suppose $x$ is any upper bound for $cA$. Then $x \leq ca \implies x/c \leq a$. This means $x/c$ is an
    upper bound of $A$. But if it's an upper bound then $x/c \geq \text{ sup } A \implies x \geq c \text{ sup }A$.

    \part $c$ inf $A$. Too lazy to write.
  \end{parts}
\end{solution}

%------------------------- Problem 6 -------------------------
\miquestion Boring. Skip.

%------------------------- Problem 7 -------------------------
\miquestion Prove that if $a$ is an upper bound for $A$, and if $a$ is also an element of $A$, then it must be that $a = $ sup $A$.
\begin{solution}
  This is basically saying $a$ is the maximum of the set.
\end{solution}

%------------------------- Problem 8 -------------------------
\miquestion Compute, without proofs, the suprema and infima (if they exist) of the following sets:
\begin{parts}
  \part \{$m/n: m, n \in \textbf{N} \text{ with } m < n$\}.
  \part \{$(-1)^{m}/n: m, n \in \textbf{N}$\}.
  \part \{$n/(3n+1): n \in \textbf{N}$\}.
  \part \{$m / (m+n): m, n \in \textbf{N}$\}.
\end{parts}

\begin{solution}
  Note that we don't include 0 in \textbf{N}.
  \begin{parts}
    \part Supremum: 1/2. Infimum: 0 (set $m=1$ and $n$ arbitrarily big).
    \part Supremum: 1. Infimum: -1.
    \part Supremum: 1/3. Infimum: 1/4.
    \part Supremum: 1/2. Infimum: 0.
  \end{parts}
\end{solution}

%------------------------- Problem 9 -------------------------
\miquestion
\begin{parts}
  \part If sup $a < $ sup $B$, show that there exists an element $b \in B$ that is an upper bound for $A$.
  \part Give an example to show that this is not always the case if we only assume sup $A \leq $ sup $B$.
\end{parts}

\begin{solution}
  \begin{parts}
    \part The idea is that there must be an element of $b$ between sup $A$ and sup $B$.
    
    Let $d = $ sup $B - $ sup $A$.
    Since sup $B$ is the *least* upper bound, then there must be an element $b \in B$ such that $b > $ sup $B - d \implies b > \text{ sup } A$.
  
    \part Let $A = [0, 1]$ and $B = [0, 1)$.
  \end{parts}
\end{solution}

%------------------------- Problem 10 -------------------------
\miquestion \textbf{Cut Property}. The Cut Property of the real numbers is the following: \\
If $A$ and $B$ are nonempty, disjoint sets with $A \cup B = \textbf{R}$ and $a < b$ for all $a \in A$ and $b \in B$, then there exists $c \in \textbf{R}$ such
that $x \leq c$ whenever $x \in A$ and $x \geq c$ whenever $x \in B$.
\begin{parts}
  \part Use the AoC to prove the Cut Property.
  \part Show that the implication goes the other way; that is, assume \textbf{R} possesses the Cut Property and let $E$ be a nonempty set that is bounded above.
  Prove sup $E$ exists.
  \part The punchline of parts (a) and (b) is that the Cut Property could be used in place of AoC as the fundamental axiom that distinguishes the real numbers from
  the rationals. To drive this point home, give a concrete example showing that the Cut Property is not a valid statement when \textbf{R} is replaced by \textbf{Q}.
\end{parts}

\begin{solution}
  \begin{parts}
    \part Any element in $b$ is an upper bound of $A$ so by AoC, $A$ has a supremum (conversely, by exercise 3, $B$ has an infimum).
    Take $c = $ sup $A$ (or $c = $ inf $B$).

    \part Since $E$ be bounded above, let $B$ be the set of all upper bounds i.e. $B = \{b \in \textbf{R}: \text{ b is an upper bound of E}\}$. But this is yet enough to invoke the cut property, since
    it's not necessarily true that $E \cup B = \textbf{R}$.

    Let $A = \{a \in \textbf{R}: a < e \text{ for some } e \in E\}$. It's easy to verify that combining $A$ and $E$ does not change the upper bounds of $E$, so let $E = A \cup E$.
    Now $E \cup B = \textbf{R}$. Invoke the Cut Property: the $c$ defined in the property is the supremum.

    \part The idea is that the "cut" point should be an irrational number. So let $A$ be all rationals less than $\sqrt{2}$ and $B$ all rationals
    greater than $\sqrt{2}$. Then $A \cup B = \textbf{Q}$ since the only missing point is not a rational number.

    Next, suppose there is a $c$ such that $x \leq c$ whenever $x \in A$. Then $c$ must be greater than $\sqrt{2}$, for if $c < \sqrt{2}$ then $c \in A$
    but if $c \in A$ then we can always find a rational greater than $c$ but less than $\sqrt{2}$. A similar argument shows that if $c$ satisfies
    $x \leq c$ for all $x \in A$ then it cannot satisfy $x \geq c$ for all $x \in B$.
  \end{parts}
\end{solution}

%------------------------- Problem 11 -------------------------
\miquestion Decide if the following statements are true or false. Give a short proof if true, or a counterexample otherwise.
\begin{parts}
  \part If $A$ and $B$ are nonempty, bounded, and satisfy $A \subseteq B$, then sup $A \leq $ sup $B$.
  \part If sup $A < $ inf $B$ for sets $A$ and $B$, then there exists a $c \in \textbf{R}$ satisfying $a < c < b$ for all $a \in A$ and $b \in B$.
  \part If there exists a $c \in \textbf{R}$ satisfying $a < c < b$ for all $a \in A$ and $b \in B$, then sup $A < $ inf $B$.
\end{parts}

\begin{solution}
  \begin{parts}
    \part True. sup $B$ is an upper bound of $A$, so the *least* upper bound of $A$ must, by definition, be less than or equal to sup $B$.
    \part True. The idea is that there must be a gap between sup $A$ and inf $B$ and we can take any number in that gap. For example, take
    $c = (\text{ sup } A + \text{ inf } B) / 2$.
    \part No. Take $A = [1, 2)$ and $B = (2, 3]$.
  \end{parts}
\end{solution}

\end{questions}
\end{document}
