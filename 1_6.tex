\documentclass{oxmathproblems}

\printanswers

\course{Understanding Analysis, 2nd Edition: Stephen Abbott}
\sheettitle{Chapter 1 - Section 1.6 - Cantor's Theorem}

\begin{document}
\begin{questions}

%------------------------- Problem 1 -------------------------
\miquestion Show that $(0, 1)$ is uncountable if and only if $R$ is uncountable.
\begin{solution}
  First we prove the necessity: if $(0, 1)$ is uncountable then $R$ is uncountable. This follows immediately from the fact that $(0, 1)$ is a subset of $R$. A superset
  of an uncountable set must be uncountable as well.

  Now the sufficiency: if $R$ is uncountable then $(0, 1)$ is uncountable. I initially wanted to prove this by contrapositive: if $(0, 1)$ is countable then $R$ is
  countable. But this is tricky. The fact that $(0, 1)$ is a subset of $R$ does not imply $R$ is countable. Then I tried using the fact that the union of (infinitely many)
  countable sets is countable, by merging infinitely many intervals. But this requires proving that the intervals other than $(0, 1)$ are also countable.

  The idea hit me to produce a bijection between $(0, 1)$ and $R$ instead, and indeed such a bijection is possible by Exercise 1.5.4. For example, the tangent function
  maps $(\frac{-\pi}{2}, \frac{\pi}{2})$ to $R$. We need just need to map $(0, 1)$ to $(\frac{-\pi}{2}, \frac{\pi}{2})$ and compose it with $tan$. Find the line equation
  joining $(0, 1)$ to $(\frac{-\pi}{2}, \frac{\pi}{2})$ and we get $\pi x - \frac{\pi}{2}$, therefore $f(x) = tan(\pi x - \frac{\pi}{2})$ is a bijection between $(0, 1)$
  and $R$. This proves both necessity and sufficiency at once.
\end{solution} 


%------------------------- Problem 2 -------------------------
\miquestion
\begin{enumerate}[label=(\alph*)]
  \item Explain why the real number $x = .b_{1}b_{2}b_{3}...$ cannot be $f(1)$.
  \item Now explain why $x \neq f(2)$, and in general why $x \neq f(n)$ for any $n \in N$.
  \item Point out the contradiction that arises from these observations and conclude that $(0, 1)$ is uncountable.
\end{enumerate}

\begin{solution}
  \begin{enumerate}[label=(\alph*)]
    \item Because $b_{1}$ is already different from $a_{11}$.
    \item In general, $b_{n} \neq a_{nn}$, so the decimal expansion differs in at least one digit.
    \item If $(0, 1)$ were countable, then $x = f(n)$ for an $n$, since $x$ is defined to be a real number $\in (0, 1)$.
    But from part (b), $x \neq f(n)$ for all $n$, a contradiction. Therefore $(0, 1)$ must be uncountable.
  \end{enumerate}
\end{solution}

%------------------------- Problem 3 -------------------------
\miquestion Supply rebuttals to the following complaints about the proof of Theorem 1.6.1 (the open interval $(0, 1)$ is uncountable).
\begin{enumerate}[label=(\alph*)]
  \item Every rational number has a decimal expansion, so we could apply the same argument to show that the set of rational numbers between 0 and 1 is
  uncountable. However, because we know that every subset of $Q$ must be uncountable, the proof of Theorem 1.6.1 must be flawed.
  
  \item Some numbers have \textit{two} different decimal representations. Specifically, any decimal expansion that terminates can also be written with
  repeating 9's. For instance, 1/2 can be written as 0.5 or as 0.49999.... Doesn't this cause some problems?
\end{enumerate}

\begin{solution}
  \begin{enumerate}[label=(\alph*)]
    \item Every rational number has a decimal expansion, but the converse is not true; not every decimal expansion represents a rational number. In the proof
    of Theorem 1.6.1, $x$ may be irrational, so it can't be used to conclude that $Q$ is uncountable.

    \item First, it's worth noting that the \textit{only way} a number can have two different decimal expansion is the one the problem statement says; one expansion that
    terminates, and another with repeating 9's. I claim this without proof, as the book also intentionally uses decimal representations without formal definitions.

    The proof of Theorem 1.6.1 works fine because it uses only the digits 2 and 3, so the counterexample it constructs is a number with a unique decimal expansion. It cannot
    be the case that $x$ is just another representation of the decimal expansion of a real number $f(n)$.
  \end{enumerate}
\end{solution}

%------------------------- Problem 4 -------------------------
\miquestion Let $S$ be the set consisting of all sequences of 0's and 1's. Observe that $S$ is not a particular sequence, but rather a large set whose elements are sequences,
namely,
\[S = \{(a_{1}, a_{2}, a_{3}, ...): a_{n} = \text{0 or 1}\}.\]

As an example, the sequence $(1, 0, 1, 0, 1, 0, ...)$ is an element of $S$, as is the sequence $(1, 1, 1, 1, 1, ...)$.
Give a rigorous argument showing that $S$ is uncountable.

\begin{solution}
  The same diagonal argument works. We just replace the decimal digits 0-9 with 0 and 1. Consider the following table like in the book
  \[
    \begin{array}{c|l}
    \mathbb{N} & (0,1) \\ \hline
    1 & f(1)=\, .\, \mathbf{a_{11}}\, a_{12}\, a_{13}\, a_{14}\, a_{15}\, a_{16}\,\cdots \\
    2 & f(2)=\, .\, a_{21}\, \mathbf{a_{22}}\, a_{23}\, a_{24}\, a_{25}\, a_{26}\,\cdots \\
    3 & f(3)=\, .\, a_{31}\, a_{32}\, \mathbf{a_{33}}\, a_{34}\, a_{35}\, a_{36}\,\cdots \\
    4 & f(4)=\, .\, a_{41}\, a_{42}\, a_{43}\, \mathbf{a_{44}}\, a_{45}\, a_{46}\,\cdots \\
    5 & f(5)=\, .\, a_{51}\, a_{52}\, a_{53}\, a_{54}\, \mathbf{a_{55}}\, a_{56}\,\cdots \\
    6 & f(6)=\, .\, a_{61}\, a_{62}\, a_{63}\, a_{64}\, a_{65}\, \mathbf{a_{66}}\,\cdots \\
    \vdots & \vdots
    \end{array}
  \]
  where $a_{mn} \in \{0, 1\}$. Construct a sequence $b$ from the diagonal such that $b_{n} = 1$ if $a_{nn} = 0$ else $b_{n} = 0$.
  Exactly the same proof as Exercise 2 works here.
\end{solution}

%------------------------- Problem 5 -------------------------
\miquestion
\begin{enumerate}[label=(\alph*)]
  \item Let $A = \{a, b, c\}$. List the eight elements of $P(A)$. (Do not forget that $\emptyset$ is considered a subset of every set).
  \item If $A$ is finite with $n$ elements, show that $P(A)$ has $2^{n}$ elements.
\end{enumerate}

\begin{solution}
  \begin{enumerate}[label=(\alph*)]
    \item $\emptyset, \{a\}, \{b\}, \{c\}, \{a, b\}, \{a, c\}, \{b, c\}, \{a, b, c\}$.
    \item For a particular subset, an element is either chosen or not chosen. So there are two choices for each of the $n$ elements, hence $2^{n}$.
  \end{enumerate}
\end{solution}

%------------------------- Problem 6 -------------------------
\miquestion
\begin{enumerate}[label=(\alph*)]
  \item Using the particular set $A = \{a, b, c\}$, exhibit two different 1-1 mappings from $A$ to $P(A)$.
  \item Letting $C = \{1, 2, 3, 4\}$, produce an example of a 1-1 map $g : C \rightarrow P(C)$.
  \item Explain why in parts (a) and (b), it's impossible to construct mappings that are \textit{onto}.
\end{enumerate}

\begin{solution}
  Note that the elements of $P(A)$ are \textit{sets}. So for example in part (a), $f(b) = b$ is not a valid mapping. It should instead be $f(b) = \{b\}$.
  \begin{enumerate}[label=(\alph*)]
    \item $f(a) = \{a\}, f(b) = \{b\}, f(c) = \{c\}$ or $f(a) = \{b\}, f(b) = \{c\}, f(c) = \{a\}$.
    \item $g(1) = \{1\}, g(2) = \{2\}, g(3) = \{3\}, g(4) = \{4\}$.
    \item Because $|P(A)| > |A|$, at least for any nonempty finite set $A$. It's also true for infinite sets,
    but that's the next exercise. 
  \end{enumerate}
\end{solution}

%------------------------- Problem 7 -------------------------
\miquestion Return to the particular functions constructed in Exercise 1.6.6 and construct the subset $B$ that results using the preceding rule (see the book).
In each case, note that $B$ is not in the range of the function used.

\begin{solution}
  We'll just do it for part (a) of Exercise 1.6.6. When the mapping is
  \[f(a) = f(a) = \{a\}, f(b) = \{b\}, f(c) = \{c\}\]
  then $B = \emptyset$, and this is outside the range of $f$.

  When the mapping is
  \[f(a) = \{b\}, f(b) = \{c\}, f(c) = \{a\}\]
  then $B = \{a, b, c\}$, and this is again outside the range of $f$.

  Also note that this is pretty much the diagonal argument. For a set $A = \{a_{1}, a_{2}, a_{3}, ...\}$ and say, $f(a_{1}) = \{a_{1}, a_{5}\}$, then draw the
  table with $f(a_{1}) = \{a_{1}, a_{2}', a_{3}', a_{4}', a_{5}, a_{6}', ...\}$ where an accent means the element is not part of $f(a_{1})$, but we list it anyway
  so the table stays rectangular. Now go through the diagonal and construct $B$ by taking accented elements.
  
  Too lazy to draw the table here. Just see the illustration in Wikipedia's page about Cantor's diagonal argument.
\end{solution}

%------------------------- Problem 8 -------------------------
\miquestion
\begin{enumerate}[label=(\alph*)]
  \item First, show that the case $a' \in B$ leads to a contradiction.
  \item Now, finish the argument by showing that the case $a' \notin B$ is equally unacceptable.
\end{enumerate}

\begin{solution}
  \begin{enumerate}[label=(\alph*)]
    \item Our assumptions are $B = f(a')$ and $a' \in B$.
    
    From the second assumption, this must mean that $a' \notin f(a')$, because that's how we constructed $B$.
    But if we have $a' \in B$ and $a' \notin f(a')$, then $B \neq f(a')$, contradicting our first assumption.

    \item Our assumptions are $B = f(a')$ and $a' \notin B$.
    
    From the second assumption, this must mean that $a' \in f(a')$, because that's how we constructed $B$.
    But if we have $a' \notin B$ and $a' \in f(a')$, then $B \neq f(a')$, contradicting our first assumption.
  \end{enumerate}
\end{solution}

\end{questions}
\end{document}
