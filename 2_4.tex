\documentclass{oxmathproblems}

\printanswers

\course{Understanding Analysis, 2nd Edition: Stephen Abbott}
\sheettitle{Chapter 2 - Section 2.4} %can leave out if no title per sheet

\begin{document}
\begin{questions}

%------------------------- Problem 1 -------------------------
\miquestion
  \begin{enumerate}[label=(\alph*)]
    \item Prove that the sequence define by $x_{1}=3$ and 
    \[x_{n+1} = \frac{1}{4-x_{n}}\]
    converges.

    \item Now that we know $\lim_{x_{n}}$ exists, explain why $\lim_{x_{n+1}}$ must also exist and equal the same value.
    \item Take the limit of each side of the recursive equation in part (a) to explicitly compute $\lim_{x_{n}}$.
  \end{enumerate}

\begin{solution}
  \begin{enumerate}[label=(\alph*)]
    \item The sequence is decreasing (TODO: prove this formally) and bounded by 3 and 0, so by the MCT it converges.
    \item The limit is the same because shifting a sequence doesn't change its limit. More formally, by using the formal definition of convergence,
    any $N$ that works for ($x_{n})$ works for $(x_{n+1})$ as well by setting $N_{2} = N - 1$.

    \item Here, $(x_{n})$ and $(x_{n+1})$ are sequences, not variables. We have
    \[\lim (x_{n+1}) = \lim({\frac{1}{4- (x_{n})}})\]
    By the algebraic limit theorem
    \[\lim (x_{n+1}) = {\frac{1}{4- \lim(x_{n})}}\]
    Let $\lim (x_{n}) = L$. By part (a) and (b)
    \[L = \frac{1}{4-L}\]
    Solve the quadratic to get $L = 2-\sqrt{3} \lor L = 2+\sqrt{3}$. Because $x_{1}=3$, the sequence is decreasing and $L = 2-sqrt{3}$.
  \end{enumerate}
\end{solution}

%------------------------- Problem 2 -------------------------
\miquestion (a) No, because the sequence doesn't converge. (b) Yes.

%------------------------- Problem 3 -------------------------
\miquestion
  \begin{enumerate}[label=(\alph*)]
    \item Show that
    \[\sqrt{2}, \sqrt{2 + \sqrt{2}}, \sqrt{2 + \sqrt{2 + \sqrt{2}}}\]
    converges and find the limit.

    \item Does the sequence
    \[\sqrt{2}, \sqrt{2\sqrt{2}}, \sqrt{2\sqrt{2\sqrt{2}}}, ...\]
    converge? If so, find the limit.
  \end{enumerate}

\begin{solution}
  \begin{enumerate}[label=(\alph*)]
    \item The sequence is $x_{n} = \sqrt{2 + x_{n-1}}$ with $x_{1} = \sqrt{2}$. It's upper bounded by 2 as $x < 2 \implies \sqrt{2+x} < (\sqrt{4}=2)$.
    
    To prove the sequence is increasing, either get the quadratic $x < \sqrt{2+x} \implies x^{2}-x-2  < 0$ and note that the starting value $x_{1}$ is
    at the increasing side of the parabola.

    Or use induction. The claim is true for $x_{1}$ and $x_{2}$. Now suppose it's true up to $x_{n-1}$. Now we need to prove $x_{n+1} > x_{n}$. Write it as
    $\sqrt{2 + x_{n-1}} > \sqrt{2 + x_{n-2}}$. By the induction hypothesis, we know $x_{n-1} > x_{n-2}$, so $\sqrt{2 + x_{n-1}} > \sqrt{2 + x_{n-2}}$ is true.

    We've proven that the sequence is increasing and is bounded by $\sqrt{2}$ and 2, so by MCT the limit exist. Find it just like problem 1.
    Let $\lim(x_{n}) = L$, then $\lim (x_{n}) = \lim {\sqrt{2 + (x_{n-1})}} \implies L = \sqrt{2 + L}$ so $L = 2$ or $L = -1$.
    As the sequence is increasing and starts from $\sqrt{2}$, $L = 2$.

    Note that the algebraic limit for sqrt has been proved at problem 2.3.1
  
    \item This is very similar to part (a). Prove the sequence is increasing, bounded by 2 (because $x < 2 \implies \sqrt{2x} < 2$), and the limit is also 2.
  \end{enumerate}
\end{solution}

%------------------------- Problem 4 -------------------------
\miquestion Skip.

%------------------------- Problem 5 -------------------------
\miquestion \textbf{Calculating Square Roots}. Let $x_{1}=2$ and define
\[x_{n+1} = \frac{1}{2}(x_{n} + \frac{2}{x_{n}})\].
  \begin{enumerate}[label=(\alph*)]
    \item Show that $x_{n}^{2}$ is $\geq$ 2, and then use this to prove that $x_{n} - x_{n+1} \geq 0$. Conclude that $\lim x_{n} = \sqrt{2}$.
    \item Modify the sequence $(x_{n})$ so that it converges to $\sqrt{c}$.
  \end{enumerate}

\begin{solution}
    \begin{enumerate}[label=(\alph*)]
      \item The claim is true for $x_{1}$. Now by induction, if $x_{n}^{2} \geq 2$, then the minimum of $x_{n+1}^{2} = \frac{1}{4}(x_{n}^{2}+4+\frac{4}{x_{n^2}})$
      is attained by substituting $x_{n}^{2} = 2 \implies \frac{1}{4}(2+4+2) = \frac{8}{4} = 2$.

      To prove the sequence is decreasing, we use induction again. $x_{1} \geq x_{2} = \frac{3}{2}$. Now we need to prove
      $x_{n+1} = \frac{1}{2}(x_{n}+\frac{2}{x_{n}}) \leq x_{n}$. Do algebra to get $x_{n}^{2} - 2 < 0 \implies x_{n}^{2} \geq 2$, which by part (a) is true.

      To confirm the limit, just do the same thing as problem 3.

      \item This is known as Heron's method. The modified sequence is $x_{n+1} = \frac{1}{2}(x_{n} + \frac{c}{x_{n}})$ with $x_{1} = c$.
      The convergence proof follows part (a) very closely.

      First we show by induction that $x_{n}^2 \geq c$ for all $n$. The claim is true for $x_{1}$. Now by induction, if $x_{n}^{2} \geq c$, then the minimum of $x_{n+1}^{2} = \frac{1}{4}(x_{n}^{2}+2c+\frac{c^2}{x_{n^2}})$
      is attained by substituting $x_{n}^{2} = c \implies \frac{1}{4}(c+2c+c) = \frac{4c}{4} = c$.

      \textbf{Important}: The solution manual seems to sweep the monotonicity proof under the rug by just saying it's similar to part (a). But it's not. This new
      sequence is \textbf{not} strictly decreasing. For example, plug $c = 0.1$, then the first two terms are increasing.

      We claim that the sequence is eventually non-increasing starting from $x_{1}$. The trick is to use the AM-GM inequality.

      First, note that $x_{n}$ is simply the arithmetic mean of $x_{n-1}$ and $\frac{c}{x_{n-1}}$. The geometric mean of these two values is $\sqrt{c}$,
      so by AM-GM we have $x_{n} \leq \sqrt{c}$. If we apply this to $n=1$, then we know $x_{1}$ is always $\geq$ $\sqrt{c}$, regardless of whether
      $x_{0}$ starts at a number larger than $\sqrt{c}$ (as in $c=2$) or less (as in $c=0.1$).

      Now we can use induction like part(a) to prove that the sequence in non-increasing from $x_{1}$ onwards. Do algebra to get $x_{n}-x_{n+1} = x_{n}^{2} - c$.
      We've proved that for $n \geq 1$, $x_{n} \geq \sqrt{c} \implies x_{n}^2 \geq c$, so $x_{n}-x_{n+1} = x_{n}^2 - c \geq 0$, so the sequence is non-increasing
      for $n \geq 1$.
    \end{enumerate}
\end{solution}

\end{questions}
\end{document}
